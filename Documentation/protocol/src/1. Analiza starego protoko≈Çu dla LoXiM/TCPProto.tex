%& --translate-file latin2pl
\documentclass[a4paper,10pt]{article}

\usepackage{polski}
\usepackage[latin2]{inputenc}

\newcommand{\nazwaprojektu}{LoXiM }
\newcommand{\nazwadokumentu}{
	Specyfikacja protoko�u komunikacyjnego\\
	Klient$\leftrightarrow$Serwer oraz Serwer$\leftrightarrow$Serwer \\
	\small{stan na 2006-10-26}}
\newcommand{\wersjadokumentu}{0.2}



% Marginesy
\textwidth17cm \textheight25.5cm \topmargin-40pt
\oddsidemargin-0.04cm


\title{\nazwaprojektu \ --- \nazwadokumentu}
\author{Piotr Tabor (pt214569@students.mimuw.edu.pl)}


\begin{document}
\maketitle

% Definicja naglowka posrodku strony
\newcommand{\nagl}[1]{\par\begin{center}{\bf{#1}}\end{center}\par}
% Definicja tego na poczatku
\newcommand{\pocz}{\par {\bf{\nazwaprojektu}}\par {\bf{\nazwadokumentu}}\par {\bf{Wersja: \wersjadokumentu}}\par}

%% Historia
% environment
\newenvironment{historia}{\nagl{Historia}\begin{tabular*}{17cm}{c|c|c|l} {\bf{Data}} & {\bf{Wersja}} & {\bf{Autor}} & {\bf{Zmiany}} \\}{\end{tabular*}}
% item
\newcommand{\hist}[4]{#1 & #2 & #3 & \parbox[t]{9cm}{#4} \\ }

%% Tabelka o znaczeniu poszczeg�lnych bajt�w w protokole
\providecommand{\znakzero}{$\backslash 000$}

\newenvironment{bajtyo}{\\
\begin{tabular*}{16.775cm}{|p{4.5cm}|p{4cm}|p{7cm}|}\hline {\bf{Od - do}} &
{\bf{Warto��}} & {\bf{Zawarto��}} \\\hline}{\hline\end{tabular*}} % item
\providecommand{\bpo}[4]{#1 $\longleftrightarrow$ #2 & #3 & \parbox[t]{7cm}{#4}
\\ }


%% Spis tresci
\newcommand{\spistresci}{\tableofcontents}


\par
{\bf{Wersja: \wersjadokumentu}}
\par

\begin{historia}
	\hist{2006-10-26}{0.1}{Piotr Tabor}{Pierwsza wersja dokumentu}
	\hist{2008-06-01}{0.2}{Piotr Tabor}{Oczyszczenie dokumentu}
\end{historia}

\spistresci
\newpage

\providecommand{\htonl}{(zast. fun. ,,htonl'')}

\section{Wst�p}
	Celem tego dokumentu jest udokumentowanie protoko�u sieciowego, jaki jest
	wykorzystywany przez semistrukturaln� baz� danych ,,LoXiM''. Konieczno��
	stworzenia tego dokumentu wynika z faktu, �e bie��cy protok� komunikacyjny nie
	jest w �aden spos�b udokumentowany, co uniemo�liwia stworzenie aplikacji integruj�cych si� z t� semistrukturaln� baz� danych. 
	
	W dokumencie pojawiaj� si� adnotacje dotycz�ce u�ycia lub jego braku - funkcji
	,,htonl''. Jest to standardowa funkcja j�zyka C, kt�ra s�u�y do konwersji
	danej liczby z architektury lokalnej komputera na liczb� kodowan� w systemie
	Big-endian.
	
	W poni�szym dokumencie pojawiaj� si� te� rozmiary i offsety wzgl�dem
	pocz�tku paczki zapisane w nawiasach zwyk�ych. U�ywa�em ich do oznaczenia
	sytuacji prawdziwej w przypadku architektury 64 bitowej (w przeciwie�stwie 
	do domy�lnie przyj�tej w tym dokumencie architektury 32bitowej). Oczywi�cie 
	wyst�puj� analogiczne r�nice dla pozosta�ych architektur - o innym rozmiarze
	s�owa procesora.    

\section{Rodzaje pakiet�w}
\begin{description}
	\item[SimpleQueryPackage] Przes�anie prostego (bezparametrowego zapytania)
	\item[ErrorPackage] Odpowied� serwera stwierdzaj�ca wyst�pienie b��du.

	\item[ParamQueryPackage]  Wys�anie zapytania, kt�re mo�e zawiera� parametry. 
	\item[StatementPackage] Otrzymanie od serwera ID wys�anego zapytania. 
	\item[ParamStatementPackage] Wys�anie parametr�w do ju� wys�anego zapytania na
	serwer. 
	
	\item[SimpleResultPackage] Odpowied� od serwera z wynikami przetwarzania
	zapytania. 
	
	\item[RemoteQueryPackage] Przetwarzanie rozproszone. Serwer prosi inny
		 serwer o wykonanie podzapytania.
	\item[RemoteResultPackage] Przetwarzanie rozproszone. Serwer odpowiada wynikami
	zapytania na zapytanie.
\end{description}

\newpage
\section{Standardowy format pakietu}

	O pakiecie jest wiadomo tylko jedno: zaczyna si� od jednobajtowej sta�ej
	�wiadcz�cej o typie pakietu.
	\begin{bajtyo}
		\bpo{0}{0}{}{Sta�a okre�laj�ca typ pakietu}	
   		\bpo{1}{n}{}{Dane pakietu}		
	\end{bajtyo}

\section{Standardowy format struktury danych}

			
		\subsection{Obiekt tekstowy = Result::STRING}
			Przesy�a �a�cuch tekstu.
			\begin{bajtyo}
               	\bpo{0}{3(7)}{Result::STRING}{Sta�a m�wi�ca o typie
               	rozwa�anego obiektu}
            	\bpo{4 (8)}			{7 (15)}		{n (unsigned long)}	{D�ugo�� stringa
            	\htonl z wliczonym znakiem $\backslash{}000$} 
   				\bpo{8 (16)}		{n+7 (n+15)} {unsigned long}			{Przesy�any �a�cuch}
 				\bpo{n+7 (n+15)}	{n+7 (n+15)}	{$\backslash{}000$} 	{Znak ko�ca �a�cucha}
		    \end{bajtyo}	
		
		\subsection{Obiekt pusty = Result::VOID}
			Dane puste - NULL.
			\begin{bajtyo}
               	\bpo{0}{3(7)}{Result::VOID}{Sta�a m�wi�ca o typie
               	rozwa�anego obiektu}            
		    \end{bajtyo}	

		\subsection{Informacja o b��dzie = Result::ERROR}
			Przysy�a unsigned long int (co z liczbami ujemnymi !!!).
			\begin{bajtyo}
               	\bpo{0}{3(7)}{Result::ERROR }{Sta�a m�wi�ca o typie
               	rozwa�anego obiektu}      
               	\bpo{4(8)}{7(15)}{unsigned long}{Numer b��du \htonl}      
		    \end{bajtyo}	

		
		\subsection{Liczba ca�kowite (dodatnia) = Result::INT}
			Przysy�a unsigned long int (co z liczbami ujemnymi !!!).
			\begin{bajtyo}
               	\bpo{0}{3(7)}{Result::INT }{Sta�a m�wi�ca o typie
               	rozwa�anego obiektu}      
               	\bpo{4(8)}{7(15)}{unsigned long}{Dana liczba \htonl}      
		    \end{bajtyo}	
						
		
		\subsection{Warto�� rzeczywista = Result::DOUBLE}
			Przesy�a liczb� rzeczywist�.  Liczba jest nie poprawnie konwertowana za
			pomoc� funkcji \htonl  			
			\begin{bajtyo}
               	\bpo{0}{3(7)}{Result::INT }{Sta�a m�wi�ca o typie
               	rozwa�anego obiektu}      
               	\bpo{4(8)}{11(15)}{double}{Dana liczba - Nie poprawne
               	u�ycie funkcji 
               	,,htonl'' - ka�de 4 bajty oddzielnie ?!}              	 
		    \end{bajtyo}	
			
		
		\subsection{Prawda = Result::BOOLTRUE}
			Typ prawdy.
			\begin{bajtyo}
               	\bpo{0}{3(7)}{Result::BOOLTRUE}{Sta�a m�wi�ca o typie
               	rozwa�anego obiektu}    
               	\bpo{4(8)}{7(15)}{1}{Sta�a m�wi�ca, �e to prawda 
               	(sam typ jak wida� nie wystarczy� autorowi tego rozwi�zania)
               	}          
		    \end{bajtyo}	
		
		\subsection{Fa�sz = Result::BOOLFALSE}
			Typ fa�szu
			\begin{bajtyo}
               	\bpo{0}{3(7)}{Result::BOOLFALSE}{Sta�a m�wi�ca o typie
               	rozwa�anego obiektu}            
             	\bpo{4(8)}{7(15)}{0}{Sta�a m�wi�ca, �e to fa�sz (sam typ jak wida� nie wystarczy� autorowi tego rozwi�zania)}             	
		    \end{bajtyo}	
		
		\subsection{Typ logiczny = Result::BOOL}
			Typ warto�ci logicznej (logika dwuwarto�ciowa)
			\begin{bajtyo}
               	\bpo{0}{3(7)}{Result::BOOL}{Sta�a m�wi�ca o typie
               	rozwa�anego obiektu}     
               	\bpo{4(8)}{7(15)}{0 lub 1}{Sta�a m�wi�ca,czy to fa�sz(0), czy prawda(1)}          
		    \end{bajtyo}	
		
		\subsection{��cznik = Result::BINDER}
		Binder - dowi�zanie pomi�dzy nazw� obiektu a warto�ci� zwi�zan� z t� nazw�. 
		\begin{bajtyo}
               	\bpo{0}{3(7)}{Result::STRING}{Sta�a m�wi�ca o typie
               	rozwa�anego obiektu}
            	\bpo{4 (8)}			{7 (15)}		{n (unsigned long)}	{D�ugo�� stringa
            	\htonl z wliczonym znakiem $\backslash{}000$} 
   				\bpo{8 (16)}		{n+7 (n+15)} {unsigned long}			{Przesy�any �a�cuch}
 				\bpo{n+7 (n+15)}	{n+7 (n+15)}	{$\backslash{}000$} 	{Znak ko�ca �a�cucha}
 				\bpo{n+8 (n+16)}	{\ldots}	{} 	{Dane zwi�zane z t� nazw�. Patrz: Standardowy
 				format struktury danych - rekurencja}
		    \end{bajtyo}	
		
		\subsection{Multizbi�r = Result::BAG}
			Implementowane przez funkcj� get/setResultCollection.
			Reprezentuje multizbi�r obiekt�w.
			\begin{bajtyo}
               	\bpo{0}{3(7)}{Result::BAG}{Sta�a m�wi�ca o typie
               	rozwa�anego obiektu}              	 
               	 \bpo{4(8)}{7(15)}{unsigned int}{Liczba m�wi�ca o liczbie obiekt�w w
		    	rozwa�anej kolekcji}
    	 	 	\bpo{8(16)}{\ldots}{}{Kolejne definicje poszczeg�lnych obiekt�w (patrz:
    	 	 	Standardowy format struktury danych - czyli rekurencja )}
		    \end{bajtyo}	
		
		\subsection{Sekwencja  = Result::SEQUENCE}	
			Tak samo jak BAG. (R�ni si� tylko identyfikatorem typu struktury danych w
			pierwszych 4 (8) bajtach: Result::SEQUENCE)	
		\subsection{Struktura  = Result::STRUCT}
			Tak samo jak BAG. (R�ni si� tylko identyfikatorem typu struktury danych w
			pierwszych 4 (8) bajtach: Result::STRUCT)	
		
		\subsection{Referencja = Result::REFERENCE}
			Referencja.
			\begin{bajtyo}
               	\bpo{0}{3(7)}{Result::REFERENCE}{Sta�a m�wi�ca o typie
               	rozwa�anego obiektu}
               	\bpo{4(8)}{7(15)}{int}{Liczba b�d�ca logicznym ID obiektu. UWAGA:
               	liczba ta obecnie nie jest poddawana konwersji
               	funkcj� ,,htonl''\ldots !!!}
		    \end{bajtyo}	
		
		\subsection{Wynik = Result::RESULT}
			Typ nie obs�ugiwany przy serializacji i deserializacji - zwraca kod b��du -2.

\newpage
\section{Przegl�d typ�w pakiet�w}
\subsection{SimpleQueryPackage}
	Pakiet tego typu s�u�y do przes�ania prostego (bezparametrowego zapytania). 	
	\begin{bajtyo}
		\bpo{0}{0}{0}{Sta�a okre�laj�ca typ pakietu}
		\bpo{1}{n}{}{�a�cuch okre�laj�cy zapytanie - nie mo�e zawiera� znaku $\backslash{}000$}		
		\bpo{n+1}{n+1}{$\backslash{}000$}{Znak ko�ca string'a $(\backslash{}000)$ - a tym samym ko�ca pakietu}
	\end{bajtyo}
	
\subsection{ErrorPackage}
	Pakiet zawiera numer b��du, kt�ry wyst�pi� po stronie serwera.
	\begin{bajtyo}
    	\bpo{0}{0}{5}{Sta�a okre�laj�ca typ pakietu}
    	\bpo{1(1)}{4(8)}{}{Numer b��du - wielko�ci sizeof(int) - a wi�c zale�ny od architektury!!!}
    \end{bajtyo}

	
\subsection{ParamQueryPackage}	
	 Wys�anie zapytania, kt�re mo�e zawiera� parametry. 
	 W strukturze komunikat nie r�ni si� niczym od SimpleQueryPackage (tzn.
	 jedyn� r�nic� jest identyfikator typu pakietu).    		
	\begin{bajtyo}
		\bpo{0}{0}{1}{Sta�a okre�laj�ca typ pakietu}
		\bpo{1}{n}{}{�a�cych okre�laj�cy zapytanie - nie mo�e zawiera� znaku $\backslash{}000$}		
		\bpo{n+1}{n+1}{$\backslash{}000$}{Znak ko�ca string'a $(\backslash{}000)$ - a tym samym ko�ca pakietu}
	\end{bajtyo}

\subsection{StatementPackage}
	 Otrzymanie od serwera ID wys�anego zapytania. 
	 \begin{bajtyo}
		\bpo{0}{0}{2}{Sta�a okre�laj�ca typ pakietu}	
   		\bpo{1}{4 (8)}{unsigned long}{Numer nadany utworzonemu zapytaniu}
	\end{bajtyo}

\subsection{ParamStatementPackage}
	Wys�anie parametr�w do ju� wys�anego zapytania na serwer. 
	\begin{bajtyo}
		\bpo{0}		{0}		{3}					{Sta�a okre�laj�ca typ pakietu}	
   		\bpo{1}		{4 (8)}	{N - unsigned long}	{Ca�kowita d�ugo�� tego pakietu}		
 		\bpo{5 (9)}	{8 (16)}{unsigned long}		{Numer ID zapytania do kt�rego przesy�amy
 		parametry}
 		\bpo{9 (17)}{12 (24)}{C - unsigned long} {Liczba parametr�w przekazywanych do
 		zapytania}
 		\bpo{13 (25)}{N}{}{C wyst�pie� bloku opisuj�cego pojedynczy parametr}		
	\end{bajtyo}

	\subsubsection{Blok opisuj�cy pojedynczy parametr}
	Pojedynczy parametr jest opisany nast�puj�c� konstrukcj�: 
	\begin{bajtyo}
   		\bpo{0}			{3 (7)}		{n (unsigned long)}	{D�ugo�� nazwy parametru (ze znakiem
   		pustym)} 
   		\bpo{4 (8)}		{n+2 (n+6)} {unsigned long}			{Nazwa parametru}
 		\bpo{n+3 (n+7)}	{n+3 (n+7)}	{$\backslash{}000$} 	{Znak ko�ca �a�cucha}
 		\bpo{n+4 (n+8)}	{\ldots}	{} 	{Definicja warto�ci (patrz: Standardowy format
 		struktury danych)} 		
	\end{bajtyo}		

\subsection{SimpleResultPackage}	
	Odpowied� od serwera z wynikami przetwarzania
	zapytania. 
	\begin{bajtyo}
		\bpo{0}{0}{4}{Sta�a okre�laj�ca typ pakietu}	
   		\bpo{1}{\ldots}{}{Definicja warto�ci (patrz: Standardowy format
 		struktury danych)}		
	\end{bajtyo}

\subsection{RemoteQueryPackage}
	Przetwarzanie rozproszone. Serwer prosi inny
 serwer o wykonanie podzapytania.
 \begin{bajtyo}
		\bpo{0}{0}{6}{Sta�a okre�laj�ca typ pakietu}	
   		\bpo{1}{1}{}{Informacja o dereferencji (0-nie,1-tak)}
   		\bpo{2}{N}{}{Remote LogicalID}		
	\end{bajtyo}

\subsection{RemoteResultPackage}
	Przetwarzanie rozproszone. Serwer odpowiada wynikami
zapytania na zapytanie.
\begin{bajtyo}
		\bpo{0}{0}{7}{Sta�a okre�laj�ca typ pakietu}	
	 	\bpo{1}{\ldots}{}{Definicja warto�ci (patrz: Standardowy format
 		struktury danych)}		
	\end{bajtyo}

\newpage
\section{Struktura plik�w}

	Wszystkie pliki bezpo�rednio zwi�zane z przesy�aniem danych przez sie�
	zawarte s� w katalogu /TCPProto	

\subsection{Tcp.h i Tcp.cpp}
	S� to pliki zawieraj�ce ,,teoretycznie'' niezale�ne funkcje
	do obs�ugi nawi�zywania po��cze� sieciowych i wysy�ania nimi danych, a tak�e
	serializowania prostych typ�w danych. Niestety istotna cz�� tych funkcji jest
	nieoprawna (nie uwa�ne stosowanie funkcji htonl - zamieniaj�cej kolejno��
	bajt�w pomi�dzy systemami opartymi o Big-endian i Little-endian).
	
	Tak�e u�yte s� nadal typu int, long - kt�rych rozmiar jest zale�ny od
	architektury na kt�rej kompilujemy system. Ponadto klasy te s� u�yte tylko w
	cz�ci przypadk�w - w pozosta�ych odczyty, konwersje i zapisy robione s�
	r�cznie. 	
	
	 
\subsection{Package.h}
	Plik Package.h zawiera deklaracj� typu abstrakcyjnego z kt�rego dziedzicz�
	wszystkie klasy do przesy�ania danych przez sie�:
	
	\begin{verbatim}
	class Package { 
    public:
    enum packageType {
        RESERVED        = -1,   //to force nondefault deserialization method
        SIMPLEQUERY = 0,   //String
        PARAMQUERY = 1,    //String with $var
        STATEMENT = 2,     //Parser tree
        PARAMSTATEMENT = 3,//stmtNr + map
        SIMPLERESULT = 4,  //Result
        ERRORRESULT = 5,   //Parse/execute error package
        REMOTEQUERY = 6,   //remote reference
        REMOTERESULT = 7   //environment section
    };
    virtual packageType getType()=0;

    //returns error code, message buffer and size of the buffer
    //it doesn't destroy the buffer
    //first byte of the buffer is the resultType
    virtual int serialize(char** buffer, int* size)=0;

    //returns error code, gets buffer and it's size
    //it destroys the buffer
    virtual int deserialize(char* buffer, int size)=0;
    virtual ~Package(){}
	};
\end{verbatim}

	Plik ten definiuje zatem identyfikatory pakiet�w zapisywane w pierwszym
	bajcie. Ponadto wymaga by klasa uto�samiana z pakietem posiada�a metod�
	umo�liwiaj�c� wczytanie pakietu z danej tablicy bajt�w (deserialize) i
	zapisanie pakietu do danej tablicy bajt�w (serialize). 
	
	Ponadto w tym pliku nag��wkowym znajduj� si� deklaracj� wszystkich pakiet�w. 

\newpage
\section{Stwierdzone wady protoko�u}

%& --translate-file latin2pl
\providecommand{\znakzero}{$\backslash 000$}

	\begin{itemize}
      \item Protok� ten nie b�dzie dzia�a� pomi�dzy komputerami r�ni�cymi
      si� architektur� lub trybem kompilacji (big endian - little 
      endian), (8, 16, 32, 64 bity).
      
      \item Brak okre�lonych z g�ry rozmiar�w pakiet�w i ich element�w - co
      utrudnia stronie odbieraj�cej alokowanie bufor�w o odpowiednich
      rozmiarach.

      \item Brak mo�liwo�ci przerwania zleconego zapytania w trakcie jego 
	wykonania     

      
      \item Ka�dy pakiet ma inn� konstrukcj� i   spos�b zapisywania tre�ci. 
      
      \item Brak negocjacji wst�pnych przy nawi�zywaniu po��czenia (logowanie,
      por�wnanie wersji, synchronizacja czasu, strona kodowa znak�w w
      �a�cuchach)
      
      \item Generalny brak konsekwencji.
      
      Czasami kod korzysta z funkcji zawartych w pliku TCPIP.cpp,a innym razem
      realizuje ca�kowicie analogiczn� operacje bezpo�rednio. 
      
      Dla przyk�adu - pola tekstowe s� przesy�ane na jeden z 3 sposob�w:
      \begin{itemize}
      	\item Jako ci�gi bajt�w, poprzedzone liczb� �wiadcz�c� o ich d�ugo�ci
      	(wliczaj�c znak \znakzero{} na ko�cu). 
      	\item Jako ci�gi bajt�w, poprzedzone liczb� �wiadcz�c� o ich d�ugo�ci
      	(nie wliczaj�c znak \znakzero{} na ko�cu).
      	\item Jako ci�gi bajt�w - z za�o�eniem, �e znak \znakzero{} ko�czy
      	przesy�any napis.
      \end{itemize}
      
      \item Brak organizacji wewn�trznej pakiet�w umo�liwiaj�cej wygodn�
      rozbudow� protoko�u (o nowe dane w poszczeg�lnych pakietach)
      
      \item Obs�uga maksymalnie maxint+1 obiekt�w (czyli np. 32768 na
      architekturach 16-bitowych). 
      
      \item Protok� nie przewiduje mo�liwo�ci przesy�ania liczb ca�kowitych
      ujemnych.
  
      \item Duplikacja kodu - wiele fragment�w kodu jest skopiowanych
      (copy-paste) z innego miejsca. Mo�na by tego unikn�� organizuj�c kod
      w odpowiedni spos�b. 
 
	  \item Istniej� trzy typy zwi�zane z przesy�aniem danych logicznych:
		Result::BOOLTRUE, Result::BOOLFALSE, Result::BOOL - ka�dy serializowalny z   
		innym identyfikatorem. Prawdopodobnie kt�ry� z nich powsta� przez pomy�k�.
    \end{itemize}


\end{document}
